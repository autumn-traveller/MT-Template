%!TEX root = ../thesis.tex

%This is the Summary
%%=========================================
\cleardoublepage
\addcontentsline{toc}{section}{Abstract}
\section*{Abstract}

The dawn of 5G brings with it an increased focus on bespoke quality of service provisions for different applications. Backhaul that respects the quality of service requirements of the applications running on-site is a difficult task, especially for remote or emergency deployments. The new space race in Lower Earth Orbit (LEO) has created a new possibility for backhaul over LEO satellite constellations. For industrial 5G sites, often located outside of cities and without access to fibre, this transforms the backhaul challenge into a multipath question.

This thesis presents an investigation into the use of multiple paths to provide deterministic backhaul in a geographically distributed 5G Campus deployment. In the course of this document a design and an implementation are presented that take a 5G flow's requirements and use knowledge collected about the available backhaul paths to calculate which path(s) to use, via a binary optimization equation. The proposed solution comprises both a control plane and a data plane component, as well as a traffic shaper. In addition, a deployment configuration designed specifically for use in geographically distributed Campus 5G Networks is described.

This thesis' solution is investigated with respect to its ability to deterministically provide various network characteristics to traffic flows which require them. This investigation is performed on a custom testbed which emulates multiple links. The results do not present conclusive evidence for the solution's ability to provide determinism, but do show promise. The latency-based path switching is 80\% optimal, and the implementation is able to reduce the experienced delay. It fails to reduce jitter for jitter-sensitive flows, and needs to compromise throughput in order to provide reduced packet loss. However it is able to provide lower packet loss than even the Optimal Single Path Oracle, via its use of path switching and packet replication.


%%=========================================
\cleardoublepage
\addcontentsline{toc}{section}{Zusammenfassung}
\section*{Zusammenfassung}

Mit der Verbreitung von 5G rückt die von speziellen Applikationen benötigte Servicequalität immer mehr in den Fokus. Insbesondere abgelegene oder für einen Notfall errichtete 5G Netze haben Probleme, einen geeigneten Backhaul bereitzustellen, der die benötigten Netzwerkeigenschaften der kritischen on-site Applikationen unterstützt. Durch die neue, stetig wachsende Verfügbarkeit der Verbindungen über LEO (Low Earth Orbit) Satellitenkonstellationen wird jetzt das traditionelle Problem vom Backhaul eine Multipath Frage.

Diese Arbeit implementiert einen Ansatz der verschiedene ausgehende Pfade benutzt und schlägt eine Lösung für die 5G Campus Umgebung vor, die für 5G Flows einen deterministischen Backhaul bereitstellt. Hierzu wird eine Architektur entworfen, die über ausgehende Pfade Daten sammelt, und anhand einer binären Optimierungsgleichung Entscheidungen trifft. Zusätzlich bietet dieser Lösungsansatz geteilte Steuerungs und Datenebenen, sowie einen Traffic Shaper, und stellt eine Konfiguration spezifisch für 5G Campus Netze vor.

In einem Testbed, dass Multipath Szenarien emuliert, wird diese Lösung, der sogenannte “WAN Connector", auf ihre deterministischen Eigenschaften untersucht. Die Ergebnisse weisen leider nach dass die Lösung keinen kompletten Determinismus anbieten kann. In der Frage der Latenz agiert die hier untersuchte und vorgeschlagene Methodik 80\% optimal, und schafft es die Latenz insgesamt zu verringern. Den Jitter zu reduzieren gelingt aber nicht, und auch der Durchsatz von Flows, die für Paketverluste und Zuverlässigkeit optimiert werden, wird extrem reduziert. Dennoch war es möglich, durch die Nutzung von Pfad-wechsel und Replikation, die Paketverluste stärker zu verringern als es mit einen Orakel der Einzelne-Pfade auswählen kann möglich gewesen wäre.