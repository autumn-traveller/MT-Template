%!TEX root = ../thesis.tex

\cleardoublepage
\chapter{Conclusion and Outlook}
\label{cha:conclusion}

%%=========================================
\section{Summary}

In review, this thesis has addressed the usage of multiple backhaul paths in a campus 5G setting to achieve determinism. A review of the topic and relevant literature was done in the background chapter, then an approach was developed based on that information and based on the requirements of the problem. Finally this approach was evaluated in a testbed which emulated multiple outgoing links, to see how it performed. The implementation provided was able to achieve some of the goals set for latency and packet loss, but struggled with jitter reduction, as well as utilizing all of the available bandwidth when it is performing packet replication to improve reliability. In the discussion of these results, their potential sources and improvements were addressed.

Placed in a wider context, this thesis presents a foray into an area which will become more and more relevant as 5G campus deployments increase and mature, and as 5G applications which require greater degrees of determinism become more commonplace. On the whole, greater degrees of bespoke packet processing are required for traffic on the internet, as application's evolve and their requirements become more strict. Especially in geographically distributed settings with multiple backhaul paths, the ability to intelligently select among these paths will be crucial to enabling these types of applications.


%%=========================================
\section{Implications and Further Areas of Research}
\label{sec:implications}

The results obtained here imply the presented approach is worthy of further investigation, however only in conjunction with some of the presented improvements. The implementation provided by this thesis cannot achieve determinism without certain adjustments, but nor can it be concluded that it will be definitively unable to do so.

Since the results were only verified in a testbed setup it would make sense to now test the WAN Connector in a real campus 5G deployment where there actually are multiple outgoing paths. Furthermore many of the improvements suggested in the previous section would all be worthy of implementing and evaluating in similar experiments.

Research should also be conducted on evaluating the performance of specific applications performance. For example, the quality of VoIP calls does'nt depend just on latency and jitter \cite{tao2005improving}, but rather how they interact together, and they have their own suites of evaluation criteria. Another specific application to consider has to be interactive video. Mission critical as well as control systems could also be considered in the test suites for evaluation.

It would be interesting to see what benefit AI and machine learning approaches may bring to this problem since they can act more dynamically, and perhaps learn the characteristics of a given link over time. Perhaps they can discover what characteristics the link exhibits right before total failure and thus perform pre-emptive path switching.
